% Created 2023-02-07 Tue 09:20
% Intended LaTeX compiler: pdflatex
\documentclass[11pt]{article}
\usepackage[utf8]{inputenc}
\usepackage[T1]{fontenc}
\usepackage{graphicx}
\usepackage{longtable}
\usepackage{wrapfig}
\usepackage{rotating}
\usepackage[normalem]{ulem}
\usepackage{amsmath}
\usepackage{amssymb}
\usepackage{capt-of}
\usepackage{hyperref}
\usepackage[slantfont,boldfont]{xeCJK}
\date{\today}
\title{}
\hypersetup{
 pdfauthor={},
 pdftitle={},
 pdfkeywords={},
 pdfsubject={},
 pdfcreator={Emacs 28.2 (Org mode 9.5.5)}, 
 pdflang={English}}
\begin{document}

\tableofcontents

应该鼓励用户哪些行为,拒绝哪些行为?如何对用户进行激励?  
鼓励作者提供优质的原创内容,鼓励读者积极参与讨论并发表更有价值的评论(点赞、转发、收藏)。拒绝作者提供低质量内容,拒绝读者低价值、无意义、影响观感的“水军”行为。  
共产/公社模型不可行,直接的市场模型也不可行,必须根据用户贡献进行奖励。如果某用户的行为得到其他用户的广泛认可,那么该行为至少具有阶段性贡献。用户得到的认可越多(声望越高),表明其对社区的贡献越大,得到的奖励应该越多。所以,可以发行一种流通通证,根据用户的声望进行奖励——声望高者奖励多。  

如何定义声望?  
声望代表用户对社区的贡献,由其他用户对该用户的认可(点赞、评论、转发、收藏、关注等)决定:  

\begin{enumerate}
\item 如果用户得到广泛的认可(被大量用户关注,文章被大量点赞、收藏、转发,评论被大量点赞、回复),那么该用户的声望应该比较高。
\item 如果用户得到平台“大V”的认可,那么该用户的声望也比较高。
\item 如果用户得到平台“大V”的唯一关注,(相比作为被该“大V”关注的用户之一,)声望要更高。
\end{enumerate}

如何计算声望?  
假设声望通过用户之间的认可进行传递(1,2),并且用户的认可所传递的声望值,与他对外所做认可的行为总量(出量)有关,出度越大,每次认可所传递的声望就越少(3)。  
根据假设,如果用户\(i\)被其他用户认可(关注、点赞、转发等),则该用户的声望  

\begin{equation}
P(i)=\sum P(j)\frac{r(i,j)}{o(j)}
\end{equation}

其中,\(r(i,j)\)为用户\(j\)对用户\(i\)的认可度,\(o(j)\)为用户\(j\)的对外总认可度。认可度根据用户的认可行为计算,不同的认可行为提供不同的认可度。例如:点赞提供1认可度,评论或回复提供10认可度等。  
声望初值如何确定?是否会影响收敛结果?  

声望通过用户之间的认可传递的依据是什么?认可行为会为被认可用户的声望背书,换言之,因为“大V”的声望高,,受“大V”认可的用户,在一定程度上也会因此受到其他用户的(非直接)认可。因此,认可的行为可以传递声望。  

为什么声望值不通过认可行为永久交付?声望的传递应该具有永续性——当我的声望高了,被我认可的用户声望也应该水涨船高。因为在我的主页可以看到所有被我认可的内容和用户,所以声望的传递应该是持续的。同时,认可行为可撤销,当认可行为撤销时,用户向被认可用户传递的这部分声望应该“回收”。并且声望需要有自然衰减机制,保证声望永远表示最近时期的用户贡献,保证奖励发放“不重不遗”。并且永久交付还需要解决声望通证的一系列,社区的总声望是固定的,还是变化的?固定的就是加直接的零和模型,很可能导致用户的认可行为大大减少。变化的就需要解决发行问题——什么情况下,社区可以发行声望?  

不活跃的用户声望会逐渐降低,合理吗?一个用户曾经对社区贡献很大,声望很高,但随着时间的推移,声望逐渐下降,这会导致老用户的流失吗?  
合理。声望不是对贡献的奖励,而是对贡献的认定。社区根据声望阶段性地进行奖励,对于曾经有过大贡献,获得高声望的用户,已经在当下就获得了奖励。也就是说,即使某用户不活跃了,他的声望下降了,也不会影响他获得应有的奖励。声望应该是一个动态的量,它总表示最近一段时间用户的贡献。所以只有持续不断地提供贡献,才能维持或提高声望。这非常符合现实——爱因斯坦为科学作出了非常大的贡献,他的贡献已经在过去就得到了认可和奖励,在当时他也因此获得了很大的声望,但今天我们不会说他声望高。我们会说郭文贵先生具有很高的声望,因为他在灭共、建立NFSC、发动爆料革命、创立G系列等事情中作出了极大的贡献。所以,声望表示的最近或长或短的一段时期用户的贡献,而过去的贡献为该用户带来的声望,会随着时间衰减,衰减的速率取决于新贡献爆发的速率和过去贡献的持续性。如果一个用户的贡献是阶段性的,那么当下他的声望可能很高,但随着时间推移,如果他没有其他贡献,声望自然逐渐衰减。如果一个用户的贡献具有持久性,里程碑意义,那么即使他的活跃度逐渐下降,没有其他贡献,只要还收益于他的贡献,许多用户还是会保持对他的认可,所以虽然他的声望会衰减,但下降的速率会更慢。甚至会出现另一种情况,就是一个很有价值的贡献,在当下并没有得到广泛的认可,该用户的声望也没有提高,但过了一段时间,大家逐渐关注并认可该用户的贡献后,他的声望也就随之变高。这意味着,即使一个用户的贡献在当下没有得到认可和奖励,也可能随着时间推移,在其他时期获得相应的认可和奖励。这是对此类贡献的补偿机制。  
除了经济奖励(分发流通通证)外,还可以设置荣誉奖励。  

能获得一定程度的认可,并且很少认可其他用户的用户很可能会聚集优势,如果他以某种方式“出售”自己的认可度,会造成严重后果吗?出于某些原因没有获得足够的认可的用户,其声望一可能比“同流合污”的“水军”还低吗?是否应该设计某种机制,以减缓优势的聚集和流失,防止声望过度倾斜,垄断奖励?  

声望的计算是否应该用户对其他用户的否定(不认可)是否应该纳入声望的计算?如果考虑否定行为,它是否会武器化,用于降低他人的声望?如果否定行为不纳入声望的计算,用户的声望是否会与其实际贡献出现较大偏差?  

如果没有新的用户贡献被认可,声望的分布是否应该随着时间的流逝在用户之间逐渐平均?如果曾经存在一批“大V”,因他们的贡献获得非常高的声望,并且得到奖励。随着时间推移,这群“大V”逐渐“泯然众人”,那么在这个过程中,他的可以得到的奖励也应该逐渐与众人持平,即使用户之间的认可没有太大的变化。否则,声望将不能体现近期的贡献,而曾经的贡献将会持续获得奖励。  

新用户如何进入声望系统?  
如何不使开通新账户、多开账户的行为得到声望激励?  

如何发行流通通证?如何根据声望分发奖励?  

如何激励那些没有得到广泛关注和认可的贡献?用户对社区的贡献越大,越有可能得到广泛的认可,声望就越高。所以,用户声望就越大,表明他的行为和提供的内容越受认可,表明他对社区的贡献越大。反之不亦然。有可能存在一些有贡献的内容和行为并不能得到广泛的关注和认可,对于提供这类内容和行为的用户,我们希望仍然可以给他们奖励,以对他们形成激励。  

此设计的优点在哪里?缺点是什么?  
\end{document}
